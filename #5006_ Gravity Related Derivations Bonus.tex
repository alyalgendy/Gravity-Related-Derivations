\documentclass{article}
\usepackage{graphicx} % Required for inserting images
\usepackage{amsmath}
 \usepackage{float}
\title{Gravity Related Derivations}
\author{Aly Algendy}
\date{July 2024}

\begin{document}

\thispagestyle{empty}

\begin{center}
    \LARGE \textbf{GRAVITY RELATED DERIVATIONS}\\
\end{center}

\begin{center}
    {\Large Stem Astronomy Club\\} 
    Academic committee\\
    Academic 23\\
    \LaTeX{} 
\end{center}
    
\vspace{55mm}

\tableofcontents

\newpage
\setcounter{page}{1}

\section{Symbols meaning}
\textit{F} is force\\
\textit{m} is mass\\
\textit{a} is acceleration\\
\textit{v} is velocity\\
\textit{t} is one cycle time\\
\textit{r} is the distance\\
\textit{G} is gravitational constant\\

\section{Kepler’s Law}

1- Newton’s Second Law
\begin{equation}
    F=ma
\end{equation}

\noindent Because the gravitational force is a centripetal one, then it will have a centripetal acceleration:
\begin{equation}
    F=m\frac{v^2}{r}
\end{equation}

\noindent since the motion of the body is circular, the covered distance in one complete cycle will be a circumference of a circle
\begin{equation}
    F=m\frac{\left(\frac{2\pi r}{t}\right)^2}{r}  
\end{equation} 

\noindent From Newton’s law of gravity
\begin{equation}
    F=G\frac{Mm}{r^2}
\end{equation}

\noindent from equation 3 and 4 we get:
\begin{equation}
\begin{aligned}
    \frac{m4\pi^2r}{t^2}&=G\frac{Mm}{r^2}\\
    \frac{4\pi^2r}{t^2}&=G\frac{M}{r^2}\\
    \frac{4\pi^2r}{t^2}&=G\frac{M}{r^2}\\
    4\pi^2r^3&=GMt^2\\
    \frac{4\pi^2r^3}{GM}&=t^2
\end{aligned}
\end{equation}

\noindent by substituting r with the semi-major a, and t with the orbital period P
\begin{equation}
    P^2=\frac{4\pi^2}{GM}a^3
\end{equation}

\noindent (Kepler’s law)

\section{Orbital Velocity}

\noindent from eq. 2 and eq. 4
\begin{equation}
\begin{aligned}
    m\frac{v^2}{r}&=G\frac{Mm}{r^2}\\
    v^2&=\frac{GM}{r}\\
    v&=\sqrt{\frac{GM}{r}}
\end{aligned}
\end{equation}

\section{Escape Velocity}

the potential energy due to gravity equals:
\begin{equation}
    E=G\frac{Mm}{r}
\end{equation}

\noindent the Kinetic energy of the body equals:
\begin{equation}
    E=\frac{1}{2}mv^2
\end{equation}

\noindent for the body to have an escape velocity (least velocity required to leave its orbit),the potential energy should equal the kinetic energy. from equations 8 and 9 we get:
\begin{equation}
    \begin{aligned}
        G\frac{Mm}{r}&=\frac{1}{2}mv^2\\
        \frac{2GM}{r}&=v^2\\
        v_{esc}&=\sqrt{\frac{2GM}{r}}
    \end{aligned}
\end{equation}

\noindent Name as in classroom: \textbf{Aly Al-Gendy}

\end{document}


uwrtdr9t 4wtiwye8 w47adoif ae7r 9adr oduo8rfdifu8r  9efidut8e ere8e
9'wept7s
dfugyrugjds8eyoftds
\theta \int_{}^{}  \,dx 